\section{Random Testing with Source Hints}
\label{sec:sources}

This section introduces the common workflow for generating random values using
automatically derived QuickCheck generators, along with its common limitations.
%
To illustrate this, let us consider the following data type definition encoding
basic HTML pages:

\begin{code}
data Html
  =  Text  String
  |  Sing  String
  |  Tag   String Html
  |  Join  Html Html
\end{code}

This type allows us to build HTML pages via four possible data constructors:
|Text| is used for plain text values; |Sing| and |Tag| represent singular and
paired HTML tags, respectively; and |Join| to simply concatenate two HTML pages
one after another.
%
Note that the constructors |Tag| and |Join| are recursive, as they have at least
one field of their own type.
%
In this paper, and in the scope of random generation in general, we will refer
to any non-recursive data constructor as \emph{terminal}, and
\emph{non-terminal} in any other case.


Then, using this representation the example page\linebreak
%
\texttt{<html>hello<hr><b>world!</b></html>}
%
can be encoded simply as:
%
\begin{code}
Tag "html" $  Text "hello"
              iJoin Single "hr"
              iJoin Tag "b" $ Text "world!"
\end{code} %$

Then, in order to generate random values of type |Html|, QuickCheck requires us
to provide a random data generator for it.

\begin{code}
class Arbitrary (a :: *) where
  arbitrary :: Gen a
\end{code}

\begin{code}
class Arbitrary (a :: *) where
  arbitrary :: Gen a
\end{code}

\begin{code}
instance Arbitrary Html where
  arbitrary = sized gen
    where
      gen 0 = frequency
        [  (2,  Text    <$> arbitrary)
        ,  (1,  Sing    <$> arbitrary) ]
      gen d = frequency
        [  (2,  Text    <$> arbitrary)
        ,  (1,  Sing    <$> arbitrary)
        ,  (4,  Tag     <$> arbitrary  <*> gen (d-1))
        ,  (3,  Join    <$> gen (d-1)  <*> gen (d-1)) ]
\end{code} %$


\begin{code}
br :: Html
br = Sing "br"

bold :: Html -> Html
bold = Tag "b"

list :: [Html] -> Html
list []  = Text "empty list"
list xs  = Tag "ul" $ foldl1 Join (Tag "li" <$> xs)

(<+>) :: Html -> Html -> Html
(<+>) x y = Join x (Join br y)
\end{code} %$

\begin{code}
  gen d = frequency
    [ ...
    , (1, pure br)
    , (5, bold   <$> gen (d-1))
    , (2, list   <$> listOf (gen (d-1)))
    , (3, (<+>)  <$> gen (d-1) <*> gen (d-1)) ]
\end{code}%$


\begin{code}
simplify :: Html -> Html
simplify (Text t1 iJoin Text t2)
  = Text (t1 ++ t2)
simplify ((Text t1 iJoin x) iJoin y)
  = simplify (Text t1 iJoin simplify (x iJoin y))
simplify (x iJoin y)
  = simplify x iJoin simplify y
simplify (Tag t x)
  = Tag t (simplify x)
simplify x = x
\end{code}

\begin{code}
  gen d = frequency
    [ ...
    , (2, do  t1  <- arbitrary
              t2  <- arbitrary
              return (Text t1 iJoin Text t2))
    , (4, do  t2  <- arbitrary
              x   <- go (d-1)
              y   <- go (d-1)
              return ((Text t1 iJoin x) iJoin y))]
\end{code}%$
