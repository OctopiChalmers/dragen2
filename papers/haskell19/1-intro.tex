\section{Introduction}
\label{sec:intro}

Pellentesque dapibus suscipit ligula. Donec posuere augue in quam. Etiam vel
tortor sodales tellus ultricies commodo. Suspendisse potenti. Aenean in sem ac
leo mollis blandit. Donec neque quam, dignissim in, mollis nec, sagittis eu,
wisi. Phasellus lacus. Etiam laoreet quam sed arcu. Phasellus at dui in ligula
mollis ultricies. Integer placerat tristique nisl. Praesent augue. Fusce
commodo. Vestibulum convallis, lorem a tempus semper, dui dui euismod elit,
vitae placerat urna tortor vitae lacus. Nullam libero mauris, consequat quis,
varius et, dictum id, arcu. Mauris mollis tincidunt felis. Aliquam feugiat
tellus ut neque. Nulla facilisis, risus a rhoncus fermentum, tellus tellus
lacinia purus, et dictum nunc justo sit amet elit.




The main contribution of this paper are:
%
\begin{itemize}
\item We identify some of the most important limitations of performing random
  testing with automatically derived generators using state-of-the-art tools
  (Section \ref{sec:sources}).
\item We present an extensible mechanism for representing random values built
  upon different source constructions using Data Types \`a la Carte (Section
  \ref{sec:representation}).
\item We develop a modular generation schema, exploiting the composable nature
  of our representations, which are extended to encode information relevant to
  the generation process at the type level (Section \ref{sec:generators}).
\item We present a very simple yet powerful domain specific language for
  describing extensible generators solely based on the types used to represent
  the desired shape of our random data (Section \ref{sec:typelevel}).
\item We provide a Template Haskell tool for automatically deriving all the
  required machinery presented throughout this paper, and evaluate its
  generation performance using three real-world case studies (Section
  \ref{sec:casestudies}).
\end{itemize}

Overall, we present a novel technique for reusing automatically derived
generators in a composable fashion, in contrast with the usual paradigm of
automatically deriving ``closed'' monolithic generators.