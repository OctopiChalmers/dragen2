%% For double-blind review submission, w/o CCS and ACM Reference (max submission space)
\documentclass[sigplan]{acmart}
\settopmatter{printfolios=true,printccs=false,printacmref=false}

% %% For double-blind review submission, w/ CCS and ACM Reference
% \documentclass[sigplan,review,anonymous]{acmart}
% \settopmatter{printfolios=true}

% %% For single-blind review submission, w/o CCS and ACM Reference (max submission space)
% \documentclass[sigplan,review]{acmart}
% \settopmatter{printfolios=true,printccs=false,printacmref=false}

% %% For single-blind review submission, w/ CCS and ACM Reference
% \documentclass[sigplan,review]{acmart}
% \settopmatter{printfolios=true}

% %% For final camera-ready submission, w/ required CCS and ACM Reference
% \documentclass[sigplan]{acmart}
% \settopmatter{}

%%%%%%%%%%%%%%%%%%%%%%%%%%%%%%%%%%%%%%%%
%% Lhs2Tex

%include polycode.fmt
%include notation.fmt

%%%%%%%%%%%%%%%%%%%%%%%%%%%%%%%%%%%%%%%%
%% Conference information
%% Supplied to authors by publisher for camera-ready submission;
%% use defaults for review submission.
\acmConference[Euro S\&P]{4th IEEE European Symposium on Security and Privacy}{June 17--19, 2019}{Stockhom, Sweden}
%\acmYear{2019}
\acmISBN{} % \acmISBN{978-x-xxxx-xxxx-x/YY/MM}
\acmDOI{} % \acmDOI{10.1145/nnnnnnn.nnnnnnn}
\startPage{1}


%%%%%%%%%%%%%%%%%%%%%%%%%%%%%%%%%%%%%%%%
%% Copyright information
%% Supplied to authors (based on authors' rights management selection;
%% see authors.acm.org) by publisher for camera-ready submission;
%% use 'none' for review submission.
\setcopyright{none}
%\setcopyright{acmcopyright}
%\setcopyright{acmlicensed}
%\setcopyright{rightsretained}
%\copyrightyear{2018}           %% If different from \acmYear


%%%%%%%%%%%%%%%%%%%%%%%%%%%%%%%%%%%%%%%%
%% Bibliography style
\bibliographystyle{ACM-Reference-Format}
%% Citation style
%\citestyle{acmauthoryear}  %% For author/year citations
%\citestyle{acmnumeric}     %% For numeric citations
%\setcitestyle{nosort}      %% With 'acmnumeric', to disable automatic
                            %% sorting of references within a single citation;
                            %% e.g., \cite{Smith99,Carpenter05,Baker12}
                            %% rendered as [14,5,2] rather than [2,5,14].
%\setcitesyle{nocompress}   %% With 'acmnumeric', to disable automatic
                            %% compression of sequential references within a
                            %% single citation;
                            %% e.g., \cite{Baker12,Baker14,Baker16}
                            %% rendered as [2,3,4] rather than [2-4].


%%%%%%%%%%%%%%%%%%%%%%%%%%%%%%%%%%%%%%%%%%%%%%%%%%%%%%%%%%%%%%%%%%%%%%
%% Note: Authors migrating a paper from traditional SIGPLAN
%% proceedings format to PACMPL format must update the
%% '\documentclass' and topmatter commands above; see
%% 'acmart-pacmpl-template.tex'.
%%%%%%%%%%%%%%%%%%%%%%%%%%%%%%%%%%%%%%%%%%%%%%%%%%%%%%%%%%%%%%%%%%%%%%


%%%%%%%%%%%%%%%%%%%%%%%%%%%%%%%%%%%%%%%%
%% Some recommended packages.
\usepackage{booktabs}
\usepackage{subcaption}
\usepackage{xcolor}
\usepackage{xspace}
\usepackage{paralist}
\usepackage{tikz}
\usepackage{pgfplots}

\usetikzlibrary{bending,patterns,matrix,fadings,positioning,shapes}
\pgfplotsset{compat=1.12}

\newcommand\quickcheck{\emph{QuickCheck}\xspace}
\newcommand\megadeth{\emph{MegaDeTH}\xspace}
\newcommand{\dragen}{DRAGEN\xspace}
\newcommand\feat{\emph{Feat}\xspace}
\newcommand\derive{\emph{derive}\xspace}



%%%%%%%%%%%%%%%%%%%%%%%%%%%%%%%%%%%%%%%%
%% The document
\begin{document}

%% Title information
\title{Automated Penetration Testing with QuickCheck}


%%%%%%%%%%%%%%%%%%%%%%%%%%%%%%%%%%%%%%%%
%% Author information
%% Contents and number of authors suppressed with 'anonymous'.
%% Each author should be introduced by \author, followed by
%% \authornote (optional), \orcid (optional), \affiliation, and
%% \email.
%% An author may have multiple affiliations and/or emails; repeat the
%% appropriate command.
%% Many elements are not rendered, but should be provided for metadata
%% extraction tools.


\author{Agust\'in Mista}
\affiliation{
  \institution{Chalmers University of Technology}
}
\email{mista@@chalmers.se}

\author{Alejandro Russo}
\affiliation{
  \institution{Chalmers University of Technology}
}
\email{russo@@chalmers.se}

%%%%%%%%%%%%%%%%%%%%%%%%%%%%%%%%%%%%%%%%
%% Abstract
%% Note: \begin{abstract}...\end{abstract} environment must come
%% before \maketitle command

\begin{abstract}
 
Random property-based testing is an increasingly popular approach to finding
bugs \cite{HughesNSA16,ArtsHNS15,HughesPAN16}.
%
In the Haskell community, \quickcheck \cite{ClaessenH00} is the dominant tool of
this sort.
%
\quickcheck requires developers to specify \emph{testing properties} describing
the expected software behavior.
%
Then, it generates a large number of random \emph{test cases} and reports those
violating the testing properties.
%
\quickcheck generates random data by employing \emph{random test data
generators} or \quickcheck generators for short.
%
The generation of test cases is guided by the \emph{types} involved in the
testing properties.
%
It defines default generators for many built-in types like booleans, integers,
and lists.
%
However, when it comes to user-defined ADTs, developers are usually required to
specify the generation process.
%
The difficulty is, however, that it might become intricate to define generators
so that they result in a suitable distribution or enforce data invariants.


Rather than manually writing generators, libraries \derive \cite{mitchell2007}
and \megadeth \cite{GriecoCB16, grieco2017} allow us to automatically synthesize generators
for a given user-defined ADT.
%
The library \derive provides no guarantees that the generation process
terminates, while \megadeth pays almost no attention to the distribution of
values.
%
In contrast, \emph{Feat} \cite{DuregardJW12} provides a mechanism to uniformly
sample values from a given ADT.
%
It enumerates all the possible values of a given ADT so that sampling uniformly
from ADTs becomes sampling uniformly from the set of natural numbers.
%---we
%
% Lastly, \emph{Luck} is a domain specific language for manually writing
% \quickcheck properties in tandem with generators so that it becomes possible to
% finely control the distribution of generated values \cite{LampropoulosGHH17}.


In this work, we consider the scenario where developers are not fully aware of
the properties and invariants that input data must fulfill.
%
This constitutes a valid assumption for \emph{penetration testing}
\cite{pentest}, where testers often apply fuzzers in an attempt to make programs
crash---an anomaly which might lead to a vulnerability.
%


Our realization is that \emph{branching processes} \cite{gw1875}, a relatively
simple stochastic model conceived to study the evolution of populations, can be
applied to predict the generation distribution of ADTs' constructors in a simple
and automatable manner.
%
To the best of our knowledge, this stochastic model has not yet been applied to
this field, and we believe it may be a promising foundation to develop future
extensions.

% , we adapt results from an area of mathematics known as
% \emph{branching processes}, and show how they help to analytically predict (at
% compile-time) the expected number of generated constructors, even in the
% presence of mutually recursive or composite ADTs.
%
Using our probabilistic formulas, we design heuristics capable of automatically
adjusting probabilities in order to synthesize generators which distributions
are aligned with users' demands \cite{DBLP:conf/haskell/MistaRH18}.

Furthermore, we provide a Haskell implementation of our mechanism in a tool 
called \dragen and perform case studies with real-world applications.
%
When generating random values, our synthesized \quickcheck generators show
improvements in code coverage when compared with those automatically derived by
state-of-the-art tools.

Recently, we extended our framework to consider addiotional sources of structural information from the users' codebase \cite{Mista2019GeneratingRS}. 
% 
In this light, our random generators can produce complex patterns of values, as well as calls to functions in abstract interfaces, obtaining remarkable improvements in terms of code coverage.
%
All of this while still being able to reason about the distributions of generated values in terms of branching processes.


% Overall, our work addresses a timely problem with a neat mathematical insight
% that is backed by a complete implementation and experience on third-party
% examples.

\end{abstract}


%%%%%%%%%%%%%%%%%%%%%%%%%%%%%%%%%%%%%%%%
%% 2012 ACM Computing Classification System (CSS) concepts
%% Generate at 'http://dl.acm.org/ccs/ccs.cfm'.
\begin{CCSXML}
<ccs2012>
<concept>
<concept_id>10011007.10011006.10011008</concept_id>
<concept_desc>Software and its engineering~General programming languages</concept_desc>
<concept_significance>500</concept_significance>
</concept>
<concept>
<concept_id>10003456.10003457.10003521.10003525</concept_id>
<concept_desc>Social and professional topics~History of programming languages</concept_desc>
<concept_significance>300</concept_significance>
</concept>
</ccs2012>
\end{CCSXML}

\ccsdesc[500]{Software and its engineering~General programming languages}
\ccsdesc[300]{Social and professional topics~History of programming languages}
%% End of generated code

%%%%%%%%%%%%%%%%%%%%%%%%%%%%%%%%%%%%%%%%
%% Keywords
%% comma separated list
%% \keywords are mandatory in final camera-ready submission
%\keywords{keyword1, keyword2, keyword3}


%%%%%%%%%%%%%%%%%%%%%%%%%%%%%%%%%%%%%%%%
%% \maketitle
%% Note: \maketitle command must come after title commands, author
%% commands, abstract environment, Computing Classification System
%% environment and commands, and keywords command.
\maketitle

%%%%%%%%%%%%%%%%%%%%%%%%%%%%%%%%%%%%%%%%
%% Sections

% Nothing here!

%%%%%%%%%%%%%%%%%%%%%%%%%%%%%%%%%%%%%%%%
% %% Acknowledgments
% \begin{acks}                            %% acks environment is optional
%                                         %% contents suppressed with 'anonymous'
%   %% Commands \grantsponsor{<sponsorID>}{<name>}{<url>} and
%   %% \grantnum[<url>]{<sponsorID>}{<number>} should be used to
%   %% acknowledge financial support and will be used by metadata
%   %% extraction tools.
%   This material is based upon work supported by the
%   \grantsponsor{GS100000001}{National Science
%     Foundation}{http://dx.doi.org/10.13039/100000001} under Grant
%   No.~\grantnum{GS100000001}{nnnnnnn} and Grant
%   No.~\grantnum{GS100000001}{mmmmmmm}.  Any opinions, findings, and
%   conclusions or recommendations expressed in this material are those
%   of the author and do not necessarily reflect the views of the
%   National Science Foundation.
% \end{acks}


%%%%%%%%%%%%%%%%%%%%%%%%%%%%%%%%%%%%%%%%
%% Bibliography
\vspace{-5pt}
\bibliography{references}

%%%%%%%%%%%%%%%%%%%%%%%%%%%%%%%%%%%%%%%%
%% That's all folks!
\end{document}
