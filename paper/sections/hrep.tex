\section{Extracting Structure} \label{sec:hrep}

In this work, we ...

\begin{code}
data HRep_Val  a = Mk_Val Int
data HRep_Add  a = Mk_Add a a
data HRep_Mul  a = Mk_Mul a a
\end{code}

For this purpose, we employ Haskell's type classes mechanism \tocite{}.
%
We define an evaluation type class $(\Downarrow)$ which specifies how to perform
a single transformation |step| from the higher-order representation back into a
concrete value of the target:

\begin{code}
class (rep down target) where
  step :: rep target -> target
\end{code}

Then, we can define the following overloaded instances of the |step| operation
for the canonical representations of data constructors simply by mapping each
lifted constructor back into its corresponding one.

\begin{code}
instance (HRep_Val down Exp) where
  step (Mk_Val n) = Val n

instance (HRep_Add down Exp) where
  step (Mk_Add x y) = Add x y

instance (HRep_Mul down Exp) where
  step (Mk_Mul x y) = Mul x y
\end{code}

With these individual representations, we can define a type combinator
$(\oplus)$ to compose two representations into a single one:

\begin{code}
data (f oplus g) a = L (f a) | R (g a)
\end{code}

Then, a composite representation can be transformed back into the concrete
target type by pattern matching on the data type variant and applying the |step|
tranformation to the inner representation:

\begin{code}
instance (f oplus g down Exp) where
  step (L f) = step f
  step (R g) = step g
\end{code}

Furthermore, we can define a type combinator ($\otimes$) to tag every
representation with an explicit generation frequency:

\begin{code}
data (f otimes n) a = Freq (f a)
\end{code}

This combinator is evaluated simply by piping the result from the inner
representation.
%
It does not change the evaluation semantics, as it is only considered at
generation time:

\begin{code}
instance (f otimes n down Exp) where
  step (Freq f) = step f
\end{code}


\begin{code}
type HRep_Exp  = HRep_Val oplus'' HRep_Add oplus'' HRep_Mul
\end{code}


\begin{code}
data (f otimes n) a = TagFreq (f a)

instance (f otimes n down Exp) where
  step (TagFreq f) = step f
\end{code}


\begin{code}
data HRep_foo_1  a = Mk_foo_1 a a
data HRep_foo_2  a = Mk_foo_2 Int a
\end{code}

\begin{code}
instance (HRep_foo_1 down Exp) where
  step (Mk_foo_1 x y)
    = Add (Add x (Val 50)) (Add (Val 25) y)

instance (HRep_foo_2 down Exp) where
  step (Mk_foo_2 x y)
    = Mul (Val 50) (Mul (Val x) y)
\end{code}

\begin{code}
data HRep_ten       a = Mk_ten
data HRep_square    a = Mk_square   a
data HRep_minus     a = Mk_minus    a a
\end{code}

\begin{code}
instance (HRep_ten down Exp) where
  step Mk_ten = ten

instance (HRep_square down Exp) where
  step (Mk_square x) = square x

instance (HRep_minus down Exp) where
  step (Mk_minus x y) = minus x y
\end{code}


\begin{code}
type HRep_foo  =       HRep_foo_1
               oplus'  HRep_foo_2
\end{code}

\begin{code}
type HRep_M  =       HRep_ten
             oplus'  HRep_square
             oplus'  HRep_minus
\end{code}

\begin{code}
type Spec_Exp  =       HRep_Exp  otimes 4
               oplus'  HRep_foo  otimes 2
               oplus'  HRep_M
\end{code}


\begin{figure}[b]
  \centering
  \section{Extracting Structure} \label{sec:hrep}

In this work, we ...

\begin{code}
data HRep_Val  a = Mk_Val Int
data HRep_Add  a = Mk_Add a a
data HRep_Mul  a = Mk_Mul a a
\end{code}

For this purpose, we employ Haskell's type classes mechanism \tocite{}.
%
We define an evaluation type class $(\Downarrow)$ which specifies how to perform
a single transformation |step| from the higher-order representation back into a
concrete value of the target:

\begin{code}
class (rep down target) where
  step :: rep target -> target
\end{code}

Then, we can define the following overloaded instances of the |step| operation
for the canonical representations of data constructors simply by mapping each
lifted constructor back into its corresponding one.

\begin{code}
instance (HRep_Val down Exp) where
  step (Mk_Val n) = Val n

instance (HRep_Add down Exp) where
  step (Mk_Add x y) = Add x y

instance (HRep_Mul down Exp) where
  step (Mk_Mul x y) = Mul x y
\end{code}

With these individual representations, we can define a type combinator
$(\oplus)$ to compose two representations into a single one:

\begin{code}
data (f oplus g) a = L (f a) | R (g a)
\end{code}

Then, a composite representation can be transformed back into the concrete
target type by pattern matching on the data type variant and applying the |step|
tranformation to the inner representation:

\begin{code}
instance (f oplus g down Exp) where
  step (L f) = step f
  step (R g) = step g
\end{code}

Furthermore, we can define a type combinator ($\otimes$) to tag every
representation with an explicit generation frequency:

\begin{code}
data (f otimes n) a = Freq (f a)
\end{code}

This combinator is evaluated back to our target data type simply by piping the
result from the inner representation.
%
It does not change the evaluation semantics, as it is only considered at
generation time:

\begin{code}
instance (f otimes n down Exp) where
  step (Freq f) = step f
\end{code}

With the introduced combinators, we can easily create a type synonym |HRep_Exp|
to refer to the canonical representation of our original data type |Exp|,
tagging for instance the representation of the constuctor |Add| to be generated
in double the proportion of rest of the representations:

\begin{code}
type HRep_Exp  =       HRep_Val
               oplus'  HRep_Add  otimes 2
               oplus'  HRep_Mul
\end{code}


\subsection{\textbf{Representing Pattern Matchings}}

\begin{code}
data HRep_foo_1  a = Mk_foo_1 a a
data HRep_foo_2  a = Mk_foo_2 Int a
\end{code}

\begin{code}
instance (HRep_foo_1 down Exp) where
  step (Mk_foo_1 x y)
    = Add (Add x (Val 50)) (Add (Val 25) y)

instance (HRep_foo_2 down Exp) where
  step (Mk_foo_2 x y)
    = Mul (Val 50) (Mul (Val x) y)
\end{code}

\begin{code}
type HRep_foo  =       HRep_foo_1
               oplus'  HRep_foo_2
\end{code}

\subsection{\textbf{Representing Abstract Interfaces}}

\begin{code}
data HRep_ten       a = Mk_ten
data HRep_square    a = Mk_square   a
data HRep_minus     a = Mk_minus    a a
\end{code}

\begin{code}
instance (HRep_ten down Exp) where
  step Mk_ten = ten

instance (HRep_square down Exp) where
  step (Mk_square x) = square x

instance (HRep_minus down Exp) where
  step (Mk_minus x y) = minus x y
\end{code}

\begin{code}
type HRep_M  =       HRep_ten
             oplus'  HRep_square
             oplus'  HRep_minus
\end{code}



\begin{code}
type Spec_Exp  =       HRep_Exp  otimes 4
               oplus'  HRep_foo  otimes 2
               oplus'  HRep_M
\end{code}

This previous definition can be interpreted graphically as it is shown in the
Figure \ref{fig:hrep}.
%
Curly arrows represent the structural information extracted using
meta-programming.

\begin{figure}[t]
  \centering
  \section{Extracting Structure} \label{sec:hrep}

In this work, we ...

\begin{code}
data HRep_Val  a = Mk_Val Int
data HRep_Add  a = Mk_Add a a
data HRep_Mul  a = Mk_Mul a a
\end{code}

For this purpose, we employ Haskell's type classes mechanism \tocite{}.
%
We define an evaluation type class $(\Downarrow)$ which specifies how to perform
a single transformation |step| from the higher-order representation back into a
concrete value of the target:

\begin{code}
class (rep down target) where
  step :: rep target -> target
\end{code}

Then, we can define the following overloaded instances of the |step| operation
for the canonical representations of data constructors simply by mapping each
lifted constructor back into its corresponding one.

\begin{code}
instance (HRep_Val down Exp) where
  step (Mk_Val n) = Val n

instance (HRep_Add down Exp) where
  step (Mk_Add x y) = Add x y

instance (HRep_Mul down Exp) where
  step (Mk_Mul x y) = Mul x y
\end{code}

With these individual representations, we can define a type combinator
$(\oplus)$ to compose two representations into a single one:

\begin{code}
data (f oplus g) a = L (f a) | R (g a)
\end{code}

Then, a composite representation can be transformed back into the concrete
target type by pattern matching on the data type variant and applying the |step|
tranformation to the inner representation:

\begin{code}
instance (f oplus g down Exp) where
  step (L f) = step f
  step (R g) = step g
\end{code}

Furthermore, we can define a type combinator ($\otimes$) to tag every
representation with an explicit generation frequency:

\begin{code}
data (f otimes n) a = Freq (f a)
\end{code}

This combinator is evaluated back to our target data type simply by piping the
result from the inner representation.
%
It does not change the evaluation semantics, as it is only considered at
generation time:

\begin{code}
instance (f otimes n down Exp) where
  step (Freq f) = step f
\end{code}

With the introduced combinators, we can easily create a type synonym |HRep_Exp|
to refer to the canonical representation of our original data type |Exp|,
tagging for instance the representation of the constuctor |Add| to be generated
in double the proportion of rest of the representations:

\begin{code}
type HRep_Exp  =       HRep_Val
               oplus'  HRep_Add  otimes 2
               oplus'  HRep_Mul
\end{code}


\subsection{\textbf{Representing Pattern Matchings}}

\begin{code}
data HRep_foo_1  a = Mk_foo_1 a a
data HRep_foo_2  a = Mk_foo_2 Int a
\end{code}

\begin{code}
instance (HRep_foo_1 down Exp) where
  step (Mk_foo_1 x y)
    = Add (Add x (Val 50)) (Add (Val 25) y)

instance (HRep_foo_2 down Exp) where
  step (Mk_foo_2 x y)
    = Mul (Val 50) (Mul (Val x) y)
\end{code}

\begin{code}
type HRep_foo  =       HRep_foo_1
               oplus'  HRep_foo_2
\end{code}

\subsection{\textbf{Representing Abstract Interfaces}}

\begin{code}
data HRep_ten       a = Mk_ten
data HRep_square    a = Mk_square   a
data HRep_minus     a = Mk_minus    a a
\end{code}

\begin{code}
instance (HRep_ten down Exp) where
  step Mk_ten = ten

instance (HRep_square down Exp) where
  step (Mk_square x) = square x

instance (HRep_minus down Exp) where
  step (Mk_minus x y) = minus x y
\end{code}

\begin{code}
type HRep_M  =       HRep_ten
             oplus'  HRep_square
             oplus'  HRep_minus
\end{code}



\begin{code}
type Spec_Exp  =       HRep_Exp  otimes 4
               oplus'  HRep_foo  otimes 2
               oplus'  HRep_M
\end{code}

This previous definition can be interpreted graphically as it is shown in the
Figure \ref{fig:hrep}.
%
Curly arrows represent the structural information extracted using
meta-programming.

\begin{figure}[t]
  \centering
  \section{Extracting Structure} \label{sec:hrep}

In this work, we ...

\begin{code}
data HRep_Val  a = Mk_Val Int
data HRep_Add  a = Mk_Add a a
data HRep_Mul  a = Mk_Mul a a
\end{code}

For this purpose, we employ Haskell's type classes mechanism \tocite{}.
%
We define an evaluation type class $(\Downarrow)$ which specifies how to perform
a single transformation |step| from the higher-order representation back into a
concrete value of the target:

\begin{code}
class (rep down target) where
  step :: rep target -> target
\end{code}

Then, we can define the following overloaded instances of the |step| operation
for the canonical representations of data constructors simply by mapping each
lifted constructor back into its corresponding one.

\begin{code}
instance (HRep_Val down Exp) where
  step (Mk_Val n) = Val n

instance (HRep_Add down Exp) where
  step (Mk_Add x y) = Add x y

instance (HRep_Mul down Exp) where
  step (Mk_Mul x y) = Mul x y
\end{code}

With these individual representations, we can define a type combinator
$(\oplus)$ to compose two representations into a single one:

\begin{code}
data (f oplus g) a = L (f a) | R (g a)
\end{code}

Then, a composite representation can be transformed back into the concrete
target type by pattern matching on the data type variant and applying the |step|
tranformation to the inner representation:

\begin{code}
instance (f oplus g down Exp) where
  step (L f) = step f
  step (R g) = step g
\end{code}

Furthermore, we can define a type combinator ($\otimes$) to tag every
representation with an explicit generation frequency:

\begin{code}
data (f otimes n) a = Freq (f a)
\end{code}

This combinator is evaluated back to our target data type simply by piping the
result from the inner representation.
%
It does not change the evaluation semantics, as it is only considered at
generation time:

\begin{code}
instance (f otimes n down Exp) where
  step (Freq f) = step f
\end{code}

With the introduced combinators, we can easily create a type synonym |HRep_Exp|
to refer to the canonical representation of our original data type |Exp|,
tagging for instance the representation of the constuctor |Add| to be generated
in double the proportion of rest of the representations:

\begin{code}
type HRep_Exp  =       HRep_Val
               oplus'  HRep_Add  otimes 2
               oplus'  HRep_Mul
\end{code}


\subsection{\textbf{Representing Pattern Matchings}}

\begin{code}
data HRep_foo_1  a = Mk_foo_1 a a
data HRep_foo_2  a = Mk_foo_2 Int a
\end{code}

\begin{code}
instance (HRep_foo_1 down Exp) where
  step (Mk_foo_1 x y)
    = Add (Add x (Val 50)) (Add (Val 25) y)

instance (HRep_foo_2 down Exp) where
  step (Mk_foo_2 x y)
    = Mul (Val 50) (Mul (Val x) y)
\end{code}

\begin{code}
type HRep_foo  =       HRep_foo_1
               oplus'  HRep_foo_2
\end{code}

\subsection{\textbf{Representing Abstract Interfaces}}

\begin{code}
data HRep_ten       a = Mk_ten
data HRep_square    a = Mk_square   a
data HRep_minus     a = Mk_minus    a a
\end{code}

\begin{code}
instance (HRep_ten down Exp) where
  step Mk_ten = ten

instance (HRep_square down Exp) where
  step (Mk_square x) = square x

instance (HRep_minus down Exp) where
  step (Mk_minus x y) = minus x y
\end{code}

\begin{code}
type HRep_M  =       HRep_ten
             oplus'  HRep_square
             oplus'  HRep_minus
\end{code}



\begin{code}
type Spec_Exp  =       HRep_Exp  otimes 4
               oplus'  HRep_foo  otimes 2
               oplus'  HRep_M
\end{code}

This previous definition can be interpreted graphically as it is shown in the
Figure \ref{fig:hrep}.
%
Curly arrows represent the structural information extracted using
meta-programming.

\begin{figure}[t]
  \centering
  \input{tikz/hrep.tex}
  \caption{Higher order representation of the data type |Exp|, using structural
    information from the function |foo| and the abstract interface of the module
    |M|.}
  \label{fig:hrep}
\end{figure}

% \begin{code}
% data (Term f) a = TagTerm (f a)

% instance (Term f down Exp) where
%   step (TagTerm f) = step f
% \end{code}
  \caption{Higher order representation of the data type |Exp|, using structural
    information from the function |foo| and the abstract interface of the module
    |M|.}
  \label{fig:hrep}
\end{figure}

% \begin{code}
% data (Term f) a = TagTerm (f a)

% instance (Term f down Exp) where
%   step (TagTerm f) = step f
% \end{code}
  \caption{Higher order representation of the data type |Exp|, using structural
    information from the function |foo| and the abstract interface of the module
    |M|.}
  \label{fig:hrep}
\end{figure}

% \begin{code}
% data (Term f) a = TagTerm (f a)

% instance (Term f down Exp) where
%   step (TagTerm f) = step f
% \end{code}
  \caption{The idea}
\end{figure}

Pellentesque dapibus suscipit ligula. Donec posuere augue in quam. Etiam vel
tortor sodales tellus ultricies commodo. Suspendisse potenti. Aenean in sem ac
leo mollis blandit. Donec neque quam, dignissim in, mollis nec, sagittis eu,
wisi. Phasellus lacus. Etiam laoreet quam sed arcu. Phasellus at dui in ligula
mollis ultricies. Integer placerat tristique nisl. Praesent augue. Fusce
commodo. Vestibulum convallis, lorem a tempus semper, dui dui euismod elit,
vitae placerat urna tortor vitae lacus. Nullam libero mauris, consequat quis,
varius et, dictum id, arcu. Mauris mollis tincidunt felis. Aliquam feugiat
tellus ut neque. Nulla facilisis, risus a rhoncus fermentum, tellus tellus
lacinia purus, et dictum nunc justo sit amet elit.

Nullam eu ante vel est convallis dignissim. Fusce suscipit, wisi nec facilisis
facilisis, est dui fermentum leo, quis tempor ligula erat quis odio. Nunc porta
vulputate tellus. Nunc rutrum turpis sed pede. Sed bibendum. Aliquam posuere.
Nunc aliquet, augue nec adipiscing interdum, lacus tellus malesuada massa, quis
varius mi purus non odio. Pellentesque condimentum, magna ut suscipit hendrerit,
ipsum augue ornare nulla, non luctus diam neque sit amet urna. Curabitur
vulputate vestibulum lorem. Fusce sagittis, libero non molestie mollis, magna
orci ultrices dolor, at vulputate neque nulla lacinia eros. Sed id ligula quis
est convallis tempor. Curabitur lacinia pulvinar nibh. Nam a sapien.

% \begin{code}
% data (Term f) a = TagTerm (f a)

% instance (Term f down Exp) where
%   step (TagTerm f) = step f
% \end{code}