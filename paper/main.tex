\documentclass[conference]{IEEEtran}

%% ODER: format ==         = "\mathrel{==}"
%% ODER: format /=         = "\neq "
%
%
\makeatletter
\@ifundefined{lhs2tex.lhs2tex.sty.read}%
  {\@namedef{lhs2tex.lhs2tex.sty.read}{}%
   \newcommand\SkipToFmtEnd{}%
   \newcommand\EndFmtInput{}%
   \long\def\SkipToFmtEnd#1\EndFmtInput{}%
  }\SkipToFmtEnd

\newcommand\ReadOnlyOnce[1]{\@ifundefined{#1}{\@namedef{#1}{}}\SkipToFmtEnd}
\usepackage{amstext}
\usepackage{amssymb}
\usepackage{stmaryrd}
\DeclareFontFamily{OT1}{cmtex}{}
\DeclareFontShape{OT1}{cmtex}{m}{n}
  {<5><6><7><8>cmtex8
   <9>cmtex9
   <10><10.95><12><14.4><17.28><20.74><24.88>cmtex10}{}
\DeclareFontShape{OT1}{cmtex}{m}{it}
  {<-> ssub * cmtt/m/it}{}
\newcommand{\texfamily}{\fontfamily{cmtex}\selectfont}
\DeclareFontShape{OT1}{cmtt}{bx}{n}
  {<5><6><7><8>cmtt8
   <9>cmbtt9
   <10><10.95><12><14.4><17.28><20.74><24.88>cmbtt10}{}
\DeclareFontShape{OT1}{cmtex}{bx}{n}
  {<-> ssub * cmtt/bx/n}{}
\newcommand{\tex}[1]{\text{\texfamily#1}}	% NEU

\newcommand{\Sp}{\hskip.33334em\relax}


\newcommand{\Conid}[1]{\mathit{#1}}
\newcommand{\Varid}[1]{\mathit{#1}}
\newcommand{\anonymous}{\kern0.06em \vbox{\hrule\@width.5em}}
\newcommand{\plus}{\mathbin{+\!\!\!+}}
\newcommand{\bind}{\mathbin{>\!\!\!>\mkern-6.7mu=}}
\newcommand{\rbind}{\mathbin{=\mkern-6.7mu<\!\!\!<}}% suggested by Neil Mitchell
\newcommand{\sequ}{\mathbin{>\!\!\!>}}
\renewcommand{\leq}{\leqslant}
\renewcommand{\geq}{\geqslant}
\usepackage{polytable}

%mathindent has to be defined
\@ifundefined{mathindent}%
  {\newdimen\mathindent\mathindent\leftmargini}%
  {}%

\def\resethooks{%
  \global\let\SaveRestoreHook\empty
  \global\let\ColumnHook\empty}
\newcommand*{\savecolumns}[1][default]%
  {\g@addto@macro\SaveRestoreHook{\savecolumns[#1]}}
\newcommand*{\restorecolumns}[1][default]%
  {\g@addto@macro\SaveRestoreHook{\restorecolumns[#1]}}
\newcommand*{\aligncolumn}[2]%
  {\g@addto@macro\ColumnHook{\column{#1}{#2}}}

\resethooks

\newcommand{\onelinecommentchars}{\quad-{}- }
\newcommand{\commentbeginchars}{\enskip\{-}
\newcommand{\commentendchars}{-\}\enskip}

\newcommand{\visiblecomments}{%
  \let\onelinecomment=\onelinecommentchars
  \let\commentbegin=\commentbeginchars
  \let\commentend=\commentendchars}

\newcommand{\invisiblecomments}{%
  \let\onelinecomment=\empty
  \let\commentbegin=\empty
  \let\commentend=\empty}

\visiblecomments

\newlength{\blanklineskip}
\setlength{\blanklineskip}{0.66084ex}

\newcommand{\hsindent}[1]{\quad}% default is fixed indentation
\let\hspre\empty
\let\hspost\empty
\newcommand{\NB}{\textbf{NB}}
\newcommand{\Todo}[1]{$\langle$\textbf{To do:}~#1$\rangle$}

\EndFmtInput
\makeatother
%
%
%
%
%
%
% This package provides two environments suitable to take the place
% of hscode, called "plainhscode" and "arrayhscode". 
%
% The plain environment surrounds each code block by vertical space,
% and it uses \abovedisplayskip and \belowdisplayskip to get spacing
% similar to formulas. Note that if these dimensions are changed,
% the spacing around displayed math formulas changes as well.
% All code is indented using \leftskip.
%
% Changed 19.08.2004 to reflect changes in colorcode. Should work with
% CodeGroup.sty.
%
\ReadOnlyOnce{polycode.fmt}%
\makeatletter

\newcommand{\hsnewpar}[1]%
  {{\parskip=0pt\parindent=0pt\par\vskip #1\noindent}}

% can be used, for instance, to redefine the code size, by setting the
% command to \small or something alike
\newcommand{\hscodestyle}{}

% The command \sethscode can be used to switch the code formatting
% behaviour by mapping the hscode environment in the subst directive
% to a new LaTeX environment.

\newcommand{\sethscode}[1]%
  {\expandafter\let\expandafter\hscode\csname #1\endcsname
   \expandafter\let\expandafter\endhscode\csname end#1\endcsname}

% "compatibility" mode restores the non-polycode.fmt layout.

\newenvironment{compathscode}%
  {\par\noindent
   \advance\leftskip\mathindent
   \hscodestyle
   \let\\=\@normalcr
   \let\hspre\(\let\hspost\)%
   \pboxed}%
  {\endpboxed\)%
   \par\noindent
   \ignorespacesafterend}

\newcommand{\compaths}{\sethscode{compathscode}}

% "plain" mode is the proposed default.
% It should now work with \centering.
% This required some changes. The old version
% is still available for reference as oldplainhscode.

\newenvironment{plainhscode}%
  {\hsnewpar\abovedisplayskip
   \advance\leftskip\mathindent
   \hscodestyle
   \let\hspre\(\let\hspost\)%
   \pboxed}%
  {\endpboxed%
   \hsnewpar\belowdisplayskip
   \ignorespacesafterend}

\newenvironment{oldplainhscode}%
  {\hsnewpar\abovedisplayskip
   \advance\leftskip\mathindent
   \hscodestyle
   \let\\=\@normalcr
   \(\pboxed}%
  {\endpboxed\)%
   \hsnewpar\belowdisplayskip
   \ignorespacesafterend}

% Here, we make plainhscode the default environment.

\newcommand{\plainhs}{\sethscode{plainhscode}}
\newcommand{\oldplainhs}{\sethscode{oldplainhscode}}
\plainhs

% The arrayhscode is like plain, but makes use of polytable's
% parray environment which disallows page breaks in code blocks.

\newenvironment{arrayhscode}%
  {\hsnewpar\abovedisplayskip
   \advance\leftskip\mathindent
   \hscodestyle
   \let\\=\@normalcr
   \(\parray}%
  {\endparray\)%
   \hsnewpar\belowdisplayskip
   \ignorespacesafterend}

\newcommand{\arrayhs}{\sethscode{arrayhscode}}

% The mathhscode environment also makes use of polytable's parray 
% environment. It is supposed to be used only inside math mode 
% (I used it to typeset the type rules in my thesis).

\newenvironment{mathhscode}%
  {\parray}{\endparray}

\newcommand{\mathhs}{\sethscode{mathhscode}}

% texths is similar to mathhs, but works in text mode.

\newenvironment{texthscode}%
  {\(\parray}{\endparray\)}

\newcommand{\texths}{\sethscode{texthscode}}

% The framed environment places code in a framed box.

\def\codeframewidth{\arrayrulewidth}
\RequirePackage{calc}

\newenvironment{framedhscode}%
  {\parskip=\abovedisplayskip\par\noindent
   \hscodestyle
   \arrayrulewidth=\codeframewidth
   \tabular{@{}|p{\linewidth-2\arraycolsep-2\arrayrulewidth-2pt}|@{}}%
   \hline\framedhslinecorrect\\{-1.5ex}%
   \let\endoflinesave=\\
   \let\\=\@normalcr
   \(\pboxed}%
  {\endpboxed\)%
   \framedhslinecorrect\endoflinesave{.5ex}\hline
   \endtabular
   \parskip=\belowdisplayskip\par\noindent
   \ignorespacesafterend}

\newcommand{\framedhslinecorrect}[2]%
  {#1[#2]}

\newcommand{\framedhs}{\sethscode{framedhscode}}

% The inlinehscode environment is an experimental environment
% that can be used to typeset displayed code inline.

\newenvironment{inlinehscode}%
  {\(\def\column##1##2{}%
   \let\>\undefined\let\<\undefined\let\\\undefined
   \newcommand\>[1][]{}\newcommand\<[1][]{}\newcommand\\[1][]{}%
   \def\fromto##1##2##3{##3}%
   \def\nextline{}}{\) }%

\newcommand{\inlinehs}{\sethscode{inlinehscode}}

% The joincode environment is a separate environment that
% can be used to surround and thereby connect multiple code
% blocks.

\newenvironment{joincode}%
  {\let\orighscode=\hscode
   \let\origendhscode=\endhscode
   \def\endhscode{\def\hscode{\endgroup\def\@currenvir{hscode}\\}\begingroup}
   %\let\SaveRestoreHook=\empty
   %\let\ColumnHook=\empty
   %\let\resethooks=\empty
   \orighscode\def\hscode{\endgroup\def\@currenvir{hscode}}}%
  {\origendhscode
   \global\let\hscode=\orighscode
   \global\let\endhscode=\origendhscode}%

\makeatother
\EndFmtInput
%






% Local Variables:
% TeX-master: "main.lhs.tex"
% TeX-command-default: "Make"
% End:

\IEEEoverridecommandlockouts
% The preceding line is only needed to identify funding in the first footnote. If that is unneeded, please comment it out.

\usepackage{todonotes}
\usepackage{cite}
\usepackage{amsmath,amssymb,amsfonts}
\usepackage{algorithmic}
\usepackage{graphicx}
\usepackage{textcomp}
\usepackage{xcolor}
\usepackage{tikz}
\usepackage{mathdots}
\usepackage{yhmath}
\usepackage{cancel}
\usepackage{color}
\usepackage{siunitx}
\usepackage{array}
\usepackage{multirow}
\usepackage{amssymb}
\usepackage{gensymb}
\usepackage{tabularx}
\usepackage{booktabs}
\usepackage{caption}
\usepackage{xspace}
\usetikzlibrary{fadings}


%%%%%%%%%%%%%%%%%%%%%%%%%%%%%%%%%%%%%%%%
%% Macros
\newcommand{\tocite}{\textbf{CITATION NEEDED}}
\newcommand{\quickcheck}{\emph{QuickCheck}\xspace}
\newcommand{\megadeth}{\emph{MegaDeTH}\xspace}
\newcommand{\dragen}{\textbf{DRAGEN}\xspace}



\begin{document}

%%%%%%%%%%%%%%%%%%%%%%%%%%%%%%%%%%%%%%%%
%% Title
\title{Synthesizing Random Generators via Higher-Order Representations}

%%%%%%%%%%%%%%%%%%%%%%%%%%%%%%%%%%%%%%%%
%% Authors
\author{
  \IEEEauthorblockN{
    % 1\textsuperscript{st}
    Agust\'in Mista
  }
  \IEEEauthorblockA{
    % \textit{Department of Computer Science and Engineering} \\
    \textit{Chalmers University of Technology}\\
    Gothenburg, Sweden \\
    mista@chalmers.se
  }
\and
\IEEEauthorblockN{
  % 2\textsuperscript{nd}
  Alejandro Russo
  }
  \IEEEauthorblockA{
    % \textit{Department of Computer Science and Engineering} \\
    \textit{Chalmers University of Technology}\\
    Gothenburg, Sweden \\
    russo@chalmers.se
  }
}

\maketitle

%%%%%%%%%%%%%%%%%%%%%%%%%%%%%%%%%%%%%%%%
%% Abstract

\begin{abstract}

  Pellentesque dapibus suscipit ligula. Donec posuere augue in quam. Etiam vel
  tortor sodales tellus ultricies commodo. Suspendisse potenti. Aenean in sem ac
  leo mollis blandit. Donec neque quam, dignissim in, mollis nec, sagittis eu,
  wisi. Phasellus lacus. Etiam laoreet quam sed arcu. Phasellus at dui in ligula
  mollis ultricies. Mauris mollis tincidunt felis. Aliquam feugiat tellus ut
  neque. Nulla facilisis, risus a rhoncus fermentum, tellus tellus lacinia
  purus, et dictum nunc justo sit amet elit.

\end{abstract}

%%%%%%%%%%%%%%%%%%%%%%%%%%%%%%%%%%%%%%%%
%% Keywords
\begin{IEEEkeywords}
component, formatting, style, styling, insert
\end{IEEEkeywords}

%%%%%%%%%%%%%%%%%%%%%%%%%%%%%%%%%%%%%%%%
%% Sections

\section{Introduction}

Pellentesque dapibus suscipit ligula. Donec posuere augue in quam. Etiam vel
tortor sodales tellus ultricies commodo. Suspendisse potenti. Aenean in sem ac
leo mollis blandit. Donec neque quam, dignissim in, mollis nec, sagittis eu,
wisi. Phasellus lacus. Etiam laoreet quam sed arcu. Phasellus at dui in ligula
mollis ultricies. Integer placerat tristique nisl. Praesent augue. Fusce
commodo. Vestibulum convallis, lorem a tempus semper, dui dui euismod elit,
vitae placerat urna tortor vitae lacus. Nullam libero mauris, consequat quis,
varius et, dictum id, arcu. Mauris mollis tincidunt felis. Aliquam feugiat
tellus ut neque. Nulla facilisis, risus a rhoncus fermentum, tellus tellus
lacinia purus, et dictum nunc justo sit amet elit.

Lorem ipsum dolor sit amet, consectetuer adipiscing elit. Donec hendrerit tempor
tellus. Donec pretium posuere tellus. Proin quam nisl, tincidunt et, mattis
eget, convallis nec, purus. Cum sociis natoque penatibus et magnis dis
parturient montes, nascetur ridiculus mus. Nulla posuere. Donec vitae dolor.
Nullam tristique diam non turpis. Cras placerat accumsan nulla. Nullam rutrum.
Nam vestibulum accumsan nisl.

Aliquam erat volutpat. Nunc eleifend leo vitae magna. In id erat non orci
commodo lobortis. Proin neque massa, cursus ut, gravida ut, lobortis eget,
lacus. Sed diam. Praesent fermentum tempor tellus. Nullam tempus. Mauris ac
felis vel velit tristique imperdiet. Donec at pede. Etiam vel neque nec dui
dignissim bibendum. Vivamus id enim. Phasellus neque orci, porta a, aliquet
quis, semper a, massa. Phasellus purus. Pellentesque tristique imperdiet tortor.
Nam euismod tellus id erat\cite{grieco2017}.

The main contribuitions of this paper are:
%
\begin{itemize}
  %
\item We identify two patological scenarios for which standard type-driven
  automatic derivation tools fail to synthesize practical random generators, due
  to a lack of either type structure or domain knowleadge (Section
  \ref{sec:randomtesting}).
  %
\item We present a generation technique able to encode stronger properties of
  the target data by reifying the static information present on the program
  codebase (Section \ref{sec:hrep}).
  %
\item We apply and extend the theory of branching processes to analitically
  predict the average distribution of generated values.
  %
  Furthermore, we use the predictions to perform simulation-based optimization
  of the random generation parameters (Section \ref{sec:synthesis}).
  %
\item We provide an implementation of our ideas in the form of a Haskell library
  to perform automatic derivation of random generators capable to extract
  useful structure information from the user source code.
  %
\end{itemize}
\section{Type-Driven Random Generation} \label{sec:randomtesting}

In this section we briefly describe the common technique used for automatically
deriving random data generators in Haskell using a type-driven approach.
%
Then, we introduce its main drawbacks by considering two scenarios where this
technique gives poor results in practice.
%
We remark that, altough this work makes focus on Haskell data types, this
technique is general enough to be applied to most programming languages with
some level of support for composite types.


Haskell is a strongly typed programming language with a powerful type system.
%
It lets programmers encode a considerable amount of information about the
properties of their systems using data types that can be checked at compilation
time.
%
One of its key aspects is the support for Algebraic Data Types (ADTs).
%
Essentially, an ADTs is a composite type defined by combining other types.
%
In the most basic conception, types can be combined by \textbf{sums} (also known
as \emph{variant types} or \emph{tagged disjoint unions}) and \textbf{products}
(or tuples) of other data types.
%
To exemplify this, consider the following definition of the data type \ensuremath{\Conid{Exp}}
encoding integer expressions:

\begin{hscode}\SaveRestoreHook
\column{B}{@{}>{\hspre}l<{\hspost}@{}}%
\column{11}{@{}>{\hspre}l<{\hspost}@{}}%
\column{15}{@{}>{\hspre}l<{\hspost}@{}}%
\column{20}{@{}>{\hspre}l<{\hspost}@{}}%
\column{25}{@{}>{\hspre}l<{\hspost}@{}}%
\column{E}{@{}>{\hspre}l<{\hspost}@{}}%
\>[B]{}\mathbf{data}\;\Conid{Exp}{}\<[11]%
\>[11]{}\mathrel{=}{}\<[15]%
\>[15]{}\Conid{Val}\;{}\<[20]%
\>[20]{}\Conid{Int}{}\<[E]%
\\
\>[11]{}\;\;\vert\;{}\<[15]%
\>[15]{}\Conid{Add}\;{}\<[20]%
\>[20]{}\Conid{Exp}\;{}\<[25]%
\>[25]{}\Conid{Exp}{}\<[E]%
\\
\>[11]{}\;\;\vert\;{}\<[15]%
\>[15]{}\Conid{Mul}\;{}\<[20]%
\>[20]{}\Conid{Exp}\;{}\<[25]%
\>[25]{}\Conid{Exp}{}\<[E]%
\ColumnHook
\end{hscode}\resethooks

In the previous definition, we declare \ensuremath{\Conid{Exp}} as the sum of three possible
classes of values: \ensuremath{\Conid{Val}} represents literal values, whereas \ensuremath{\Conid{Add}} and \ensuremath{\Conid{Mul}}
represent the addition and multiplication of two integer expressions,
respectively.
%
In Haskell, \ensuremath{\Conid{Val}}, \ensuremath{\Conid{Add}} and \ensuremath{\Conid{Mul}} are called \emph{data constructors} and are
used to distiguish which variant of the data type we are reffering to.
%
Each data constructor is then defined as a product of zero or more types known
as \emph{fields}.
%
In particular, \ensuremath{\Conid{Val}} contains a field of type \ensuremath{\Conid{Int}}, while \ensuremath{\Conid{Add}} and \ensuremath{\Conid{Mul}}
contain two fields of type \ensuremath{\Conid{Exp}} representing the operands of each operation.
%
Note that \ensuremath{\Conid{Exp}} is used as a field of at least one of its data constructors
(case \ensuremath{\Conid{Add}} and \ensuremath{\Conid{Mul}}), making it a recursively defined ADT.
%
In general, we will say that a data constructor with no recursive fields is
\emph{terminal}, and \emph{non-terminal} or \emph{recursive} if it has at least
one field of such nature.
%
With this reprentation, the expression ``2 + (5 * 6)'' can be encoded simply by
using \ensuremath{\Conid{Exp}} data constructors as \ensuremath{(\Conid{Add}\;(\Conid{Val}\;\mathrm{2})\;(\Conid{Mul}\;(\Conid{Val}\;\mathrm{5})\;(\Conid{Val}\;\mathrm{6})))}.
%
Furthermore, an evaluation function from \ensuremath{\Conid{Exp}}s to integer values can be defined
very idiomatically:

\begin{hscode}\SaveRestoreHook
\column{B}{@{}>{\hspre}l<{\hspost}@{}}%
\column{17}{@{}>{\hspre}l<{\hspost}@{}}%
\column{27}{@{}>{\hspre}c<{\hspost}@{}}%
\column{27E}{@{}l@{}}%
\column{30}{@{}>{\hspre}l<{\hspost}@{}}%
\column{E}{@{}>{\hspre}l<{\hspost}@{}}%
\>[B]{}\Varid{eval}\mathbin{::}\Conid{Exp}\to \Conid{Int}{}\<[E]%
\\
\>[B]{}\Varid{eval}\;(\Conid{Val}\;\Varid{n}){}\<[17]%
\>[17]{}\mathrel{=}\Varid{n}{}\<[E]%
\\
\>[B]{}\Varid{eval}\;(\Conid{Add}\;\Varid{x}\;\Varid{y}){}\<[17]%
\>[17]{}\mathrel{=}\Varid{eval}\;\Varid{x}{}\<[27]%
\>[27]{}\mathbin{+}{}\<[27E]%
\>[30]{}\Varid{eval}\;\Varid{y}{}\<[E]%
\\
\>[B]{}\Varid{eval}\;(\Conid{Mul}\;\Varid{x}\;\Varid{y}){}\<[17]%
\>[17]{}\mathrel{=}\Varid{eval}\;\Varid{x}{}\<[27]%
\>[27]{}\mathbin{*}{}\<[27E]%
\>[30]{}\Varid{eval}\;\Varid{y}{}\<[E]%
\ColumnHook
\end{hscode}\resethooks

In the previous definition, \ensuremath{\Varid{eval}} is described using \emph{pattern matching}
over each possible kind of value.
%
For the input case of a literal values, we simply return the value contained in
the \ensuremath{\Conid{Val}} constructor.
%
On the other hand, if the input value matches either an \ensuremath{\Conid{Add}} or a \ensuremath{\Conid{Mul}} data
constructor, where \ensuremath{\Varid{x}} and \ensuremath{\Varid{y}} are \emph{pattern variables} that match any
subexpression, we recursively evaluate these subexpressions, combining them with
the apropiate operation on each case.

\subsection{\textbf{Type-Driven Generation of Random Values}}

In order to perform random testing of a Haskell codebase involving user defined
data types, most approaches require the user to provide a random data generator
for each one of them.
%
This tends to be a cumbersome and error prone task that closely follows the data
type structure.
%
For instance, consider the following definition of a \quickcheck random
generator for the type \ensuremath{\Conid{Exp}}:

\begin{hscode}\SaveRestoreHook
\column{B}{@{}>{\hspre}l<{\hspost}@{}}%
\column{3}{@{}>{\hspre}l<{\hspost}@{}}%
\column{5}{@{}>{\hspre}l<{\hspost}@{}}%
\column{7}{@{}>{\hspre}c<{\hspost}@{}}%
\column{7E}{@{}l@{}}%
\column{10}{@{}>{\hspre}l<{\hspost}@{}}%
\column{15}{@{}>{\hspre}l<{\hspost}@{}}%
\column{30}{@{}>{\hspre}l<{\hspost}@{}}%
\column{E}{@{}>{\hspre}l<{\hspost}@{}}%
\>[B]{}\Varid{genExp}\mathbin{::}\Conid{Gen}\;\Conid{Exp}{}\<[E]%
\\
\>[B]{}\Varid{genExp}\mathrel{=}\Varid{sized}\;\Varid{gen}{}\<[E]%
\\
\>[B]{}\hsindent{3}{}\<[3]%
\>[3]{}\mathbf{where}{}\<[E]%
\\
\>[3]{}\hsindent{2}{}\<[5]%
\>[5]{}\Varid{gen}\;\mathrm{0}\mathrel{=}\Conid{Val}\mathop{\langle \$ \rangle}\Varid{arbitrary}{}\<[E]%
\\
\>[3]{}\hsindent{2}{}\<[5]%
\>[5]{}\Varid{gen}\;\Varid{n}\mathrel{=}\Varid{oneof}{}\<[E]%
\\
\>[5]{}\hsindent{2}{}\<[7]%
\>[7]{}[\mskip1.5mu {}\<[7E]%
\>[10]{}\Conid{Val}{}\<[15]%
\>[15]{}\mathop{\langle \$ \rangle}\Varid{arbitrary}{}\<[E]%
\\
\>[5]{}\hsindent{2}{}\<[7]%
\>[7]{},{}\<[7E]%
\>[10]{}\Conid{Add}{}\<[15]%
\>[15]{}\mathop{\langle \$ \rangle}\Varid{gen}\;(n{-}1){}\<[30]%
\>[30]{}\mathop{\langle \ast \rangle}\Varid{gen}\;(n{-}1){}\<[E]%
\\
\>[5]{}\hsindent{2}{}\<[7]%
\>[7]{},{}\<[7E]%
\>[10]{}\Conid{Mul}{}\<[15]%
\>[15]{}\mathop{\langle \$ \rangle}\Varid{gen}\;(n{-}1){}\<[30]%
\>[30]{}\mathop{\langle \ast \rangle}\Varid{gen}\;(n{-}1)\mskip1.5mu]{}\<[E]%
\ColumnHook
\end{hscode}\resethooks
 %$

This random generator is defined using \quickcheck's combinator \ensuremath{\Varid{sized}\mathbin{::}(\Conid{Int}\to \Conid{Gen}\;\Varid{a})\to \Conid{Gen}\;\Varid{a}} to paremetrize the generation process up to an external
integer number known as the \emph{generation size}.
%
This parameter is used to limit the maximum amount of recursive call that this
random generator can perform.
%
When called with a positive generation size (case \ensuremath{\Varid{gen}\;\Varid{n}}), the generator is free
to randomly pick with uniform probability among any data constructor of \ensuremath{\Conid{Exp}}.
%
In the case it picks to generate a \ensuremath{\Conid{Val}} data constructor, it also generates a
random \ensuremath{\Conid{Int}} value using the standard overloaded generator \ensuremath{\Varid{arbitrary}}
(\quickcheck provides standard random generators for most base types like \ensuremath{\Conid{Int}},
\ensuremath{\Conid{Bool}}, etc.).
%
On the other hand, when it randomly picks to generate either an \ensuremath{\Conid{Add}} or a \ensuremath{\Conid{Mul}}
data constructor, it also generates independently two random subexpressions
corresponding to the data constructor fields by calling itself recursively (\ensuremath{\Varid{gen}\;(n{-}1)}), ensuring to decrease the generation size by a unit on each recursive
invocation.


This procedure keeps calling itself recursively until the generation size
reaches zero (case \ensuremath{\Varid{gen}\;\mathrm{0}}), where the generator is constrained to pick only
among terminal data constructors, being \ensuremath{\Conid{Val}} the only possible choice in our
particular case.
%
Strictly decreasing the the generation size by one on each recursive call
results in a useful property over the generated data: every generated value has
at most \ensuremath{\Varid{n}} levels, where \ensuremath{\Varid{n}} is the generation size set by the user.
%
This property enables us to model the generation process using the theory of
branching processes introduced by \tocite, and extended in this work as
described in Section \ref{sec:synthesis}.


Having discussed the previous random generator definition, it easy to understand
how to extend this generation procedure to any data type defined in an algebraic
fashion.
%
Fortunately, there exists different meta-programming tools to avoid the user
from having to mechanically write random generators over and over again for each
user-defined ADT.
%
The simplest tool for such purpose is \megadeth.
%
Given the name of the target data type, it synthesizes a random generator for it
that behaves pretty much like the previously defined one.
%
However, picking among different data constructors with uniform probability can
lead to a generation process biased to generate (in average) very small values,
regardless of the generation sized set by the user \tocite.

\dragen is meta-programming tool conceived to mitigate this problem.
%
Instead of setting a uniform generation probability of data constructors, this
tool considers the scenario where each one can be generated following a
different (although fixed over time) generation probability.
%
Then, this tool uses the theory of branching processes to analitically predict
the average distribution of data constructors generated on each random value.
%
This prediction mechanism is used to feedback a simulation-based optimization
process that adjusts the generation probability of each data constructor in
order to obtain a particular distribution of values that can be specified by the
user, offering a more flexible testing environment while still being mostly
automated.


Both \megadeth and \dragen synthesize random generators that are theoretically
capable to generate the whole space of values of the target data type, provided
that we set a generation size suficiently big.
%
However, the limitations arise quickly when we consider that the underlying
generation model is essentially the same: they pick a single data constructor
and recusively generate each required subexpression independently.
%
In practice, this procedure is too generic to generate complex data that is
useful enough to be used for random testing.


In this work we identify two sources of complications that appear when using the
aforementioned automatic derivation tools to obtain random data generators:

\begin{enumerate}
\item The target code behaves differently over inputs matching very specific
  patterns.
\item The target code behaves differently according to some dependency over
  subexpressions of its inputs.
\end{enumerate}

To exemplify the first point, suppose we want to use randomly generated \ensuremath{\Conid{Exp}}s
to test a property comprising the following function:

\begin{hscode}\SaveRestoreHook
\column{B}{@{}>{\hspre}l<{\hspost}@{}}%
\column{3}{@{}>{\hspre}l<{\hspost}@{}}%
\column{E}{@{}>{\hspre}l<{\hspost}@{}}%
\>[B]{}\Varid{foo}\mathbin{::}\Conid{Exp}\to \Conid{Exp}{}\<[E]%
\\
\>[B]{}\Varid{foo}\;(\Conid{Add}\;(\Conid{Add}\;\Varid{x}\;(\Conid{Val}\;\mathrm{50}))\;(\Conid{Add}\;(\Conid{Val}\;\mathrm{25})\;\Varid{y})){}\<[E]%
\\
\>[B]{}\hsindent{3}{}\<[3]%
\>[3]{}\mathrel{=}\Varid{error}\;\text{\ttfamily \char34 pattern~\#1\char34}{}\<[E]%
\\
\>[B]{}\Varid{foo}\;(\Conid{Mul}\;(\Conid{Mul}\;\Varid{x}\;(\Conid{Val}\;\mathrm{50}))\;(\Conid{Mul}\;(\Conid{Val}\;\mathrm{25})\;\Varid{y})){}\<[E]%
\\
\>[B]{}\hsindent{3}{}\<[3]%
\>[3]{}\mathrel{=}\Varid{error}\;\text{\ttfamily \char34 pattern~\#2\char34}{}\<[E]%
\\
\>[B]{}\Varid{foo}\;\Varid{x}\mathrel{=}\Varid{x}{}\<[E]%
\ColumnHook
\end{hscode}\resethooks

This function behaves mostly like an identity function, except for two very
specific patterns of input values defined using three nested levels of pattern
matching.
%
This is, \ensuremath{\Varid{foo}} does not only matches against the root data constructor, but also
against the data constructors of its subexpressions and sub-subexpressions.

If we consider that \quickcheck's default implementation for random \ensuremath{\Conid{Int}}s pick
a number in the interval $[-n, n]$ with uniform distribution, we need to be able
to identify which one is a sensible generation size to set based on wether our
code branches on numbers or not, otherwise our random generator will not be able
to generate random values to match against such numbers, leaving fragments of
code completely untested.
%
Suppose that we pick the a generation size $50$, i.e. the minimum size that is
big enough to produce an \ensuremath{\Conid{Int}} number equal to $50$.
%
Under this considerations, the probability of generating a value matching the
first clause of \ensuremath{\Varid{foo}} (and hence triggering the error) are as follows:

\begin{align*}
  P(match(foo_1))
  \!&&=\; &P(Add) \\
  \!&&*\; &P(Add) * 1 * (P(Val) * (1/100)) \\
  \!&&*\; &P(Add) * (P(Val) * (1/100)) * 1
\end{align*}

\todo[inline]{AM: COMPLETE}


Despite the fact that the previous example might look artificial, branching
against specific patterns of the input data is a common task.
%
For instance, the balancing function of a Red-Black tree need to consider a
quite specific combination of color, left- and right subtrees and sub-subtrees
in order to preserve the height invariant.
%
Moreover, Common Subexpression Elimination (CSE) is a compiler optimization that
needs to consider very specific sequences of instructions that may be regrouped
in a computationally more efficient way.

We believe that if a certain pattern of values is relevant enough to appear in
the codebase being tested, a practical random testing methodology should be able
to produce values satisfying such pattern in a substantial proportion.


\begin{hscode}\SaveRestoreHook
\column{B}{@{}>{\hspre}l<{\hspost}@{}}%
\column{E}{@{}>{\hspre}l<{\hspost}@{}}%
\>[B]{}\Varid{fortyTwo}\mathbin{::}\Conid{Exp}{}\<[E]%
\\
\>[B]{}\Varid{fortyTwo}\mathrel{=}\Conid{Val}\;\mathrm{42}{}\<[E]%
\\[\blanklineskip]%
\>[B]{}\Varid{twice}\mathbin{::}\Conid{Exp}\to \Conid{Exp}{}\<[E]%
\\
\>[B]{}\Varid{twice}\;\Varid{x}\mathrel{=}\Conid{Add}\;\Varid{x}\;\Varid{x}{}\<[E]%
\\[\blanklineskip]%
\>[B]{}\Varid{square}\mathbin{::}\Conid{Exp}\to \Conid{Exp}{}\<[E]%
\\
\>[B]{}\Varid{square}\;\Varid{x}\mathrel{=}\Conid{Mul}\;\Varid{x}\;\Varid{x}{}\<[E]%
\\[\blanklineskip]%
\>[B]{}\Varid{neg}\mathbin{::}\Conid{Exp}\to \Conid{Exp}{}\<[E]%
\\
\>[B]{}\Varid{neg}\;\Varid{x}\mathrel{=}\Conid{Mul}\;\Varid{x}\;(\Conid{Val}\;(\mathbin{-}\mathrm{1})){}\<[E]%
\\[\blanklineskip]%
\>[B]{}\Varid{minus}\mathbin{::}\Conid{Exp}\to \Conid{Exp}\to \Conid{Exp}{}\<[E]%
\\
\>[B]{}\Varid{minus}\;\Varid{x}\;\Varid{y}\mathrel{=}\Conid{Add}\;\Varid{x}\;(\Varid{neg}\;\Varid{y}){}\<[E]%
\ColumnHook
\end{hscode}\resethooks


% \begin{code}
% data IExp  =   VAL Int
%            pp  ADD IExp IExp
%            pp  IF BExp IExp IExp

% data BExp  =   TRUE | FALSE
%            pp  LE IExp IExp
%            pp  AND BExp BExp
% \end{code}

% \begin{code}
% foo :: IExp -> IExp
% foo (ADD (IF TRUE (VAL 42) x) y)     = ...
% foo (IF (LE (VAL 24) x) (VAL y)) z)  = ...
% foo x                                = ...

% bar :: BExp -> BExp
% bar (LE (ADD (VAL 12) x) (VAL 34))   = ...
% bar (LE (VAL 78) (IF FALSE x y)) z)  = ...
% bar x                                = ...
% \end{code}

% \begin{code}
% twice :: IExp -> IExp
% twice e = ADD e e

% equals :: IExp -> IExp -> BExp
% equals x y = AND (LE x y) (LE y x)

% and3 :: BExp -> BExp -> BExp -> BExp
% and3 x y z = AND x (AND y z)
% \end{code}
\section{Extracting Structure} \label{sec:hrep}

What to say here:

\begin{itemize}
\item Haskell ADTs
\item Type driven generation
\item Limitations
\item Key idea: Higher order representations
\item Canonical representation of the data type
\item Pattern matchings representation
\item Abstract interface representation
\item Composing representations
\end{itemize}

\begin{figure}[b]
  \centering
  \section{Extracting Structure} \label{sec:hrep}

In this work, we ...

\begin{code}
data HRep_Val  a = Mk_Val Int
data HRep_Add  a = Mk_Add a a
data HRep_Mul  a = Mk_Mul a a
\end{code}

For this purpose, we employ Haskell's type classes mechanism \tocite{}.
%
We define an evaluation type class $(\Downarrow)$ which specifies how to perform
a single transformation |step| from the higher-order representation back into a
concrete value of the target:

\begin{code}
class (rep down target) where
  step :: rep target -> target
\end{code}

Then, we can define the following overloaded instances of the |step| operation
for the canonical representations of data constructors simply by mapping each
lifted constructor back into its corresponding one.

\begin{code}
instance (HRep_Val down Exp) where
  step (Mk_Val n) = Val n

instance (HRep_Add down Exp) where
  step (Mk_Add x y) = Add x y

instance (HRep_Mul down Exp) where
  step (Mk_Mul x y) = Mul x y
\end{code}

With these individual representations, we can define a type combinator
$(\oplus)$ to compose two representations into a single one:

\begin{code}
data (f oplus g) a = L (f a) | R (g a)
\end{code}

Then, a composite representation can be transformed back into the concrete
target type by pattern matching on the data type variant and applying the |step|
tranformation to the inner representation:

\begin{code}
instance (f oplus g down Exp) where
  step (L f) = step f
  step (R g) = step g
\end{code}

Furthermore, we can define a type combinator ($\otimes$) to tag every
representation with an explicit generation frequency:

\begin{code}
data (f otimes n) a = Freq (f a)
\end{code}

This combinator is evaluated back to our target data type simply by piping the
result from the inner representation.
%
It does not change the evaluation semantics, as it is only considered at
generation time:

\begin{code}
instance (f otimes n down Exp) where
  step (Freq f) = step f
\end{code}

With the introduced combinators, we can easily create a type synonym |HRep_Exp|
to refer to the canonical representation of our original data type |Exp|,
tagging for instance the representation of the constuctor |Add| to be generated
in double the proportion of rest of the representations:

\begin{code}
type HRep_Exp  =       HRep_Val
               oplus'  HRep_Add  otimes 2
               oplus'  HRep_Mul
\end{code}


\subsection{\textbf{Representing Pattern Matchings}}

\begin{code}
data HRep_foo_1  a = Mk_foo_1 a a
data HRep_foo_2  a = Mk_foo_2 Int a
\end{code}

\begin{code}
instance (HRep_foo_1 down Exp) where
  step (Mk_foo_1 x y)
    = Add (Add x (Val 50)) (Add (Val 25) y)

instance (HRep_foo_2 down Exp) where
  step (Mk_foo_2 x y)
    = Mul (Val 50) (Mul (Val x) y)
\end{code}

\begin{code}
type HRep_foo  =       HRep_foo_1
               oplus'  HRep_foo_2
\end{code}

\subsection{\textbf{Representing Abstract Interfaces}}

\begin{code}
data HRep_ten       a = Mk_ten
data HRep_square    a = Mk_square   a
data HRep_minus     a = Mk_minus    a a
\end{code}

\begin{code}
instance (HRep_ten down Exp) where
  step Mk_ten = ten

instance (HRep_square down Exp) where
  step (Mk_square x) = square x

instance (HRep_minus down Exp) where
  step (Mk_minus x y) = minus x y
\end{code}

\begin{code}
type HRep_M  =       HRep_ten
             oplus'  HRep_square
             oplus'  HRep_minus
\end{code}



\begin{code}
type Spec_Exp  =       HRep_Exp  otimes 4
               oplus'  HRep_foo  otimes 2
               oplus'  HRep_M
\end{code}

This previous definition can be interpreted graphically as it is shown in the
Figure \ref{fig:hrep}.
%
Curly arrows represent the structural information extracted using
meta-programming.

\begin{figure}[t]
  \centering
  \section{Extracting Structure} \label{sec:hrep}

In this work, we ...

\begin{code}
data HRep_Val  a = Mk_Val Int
data HRep_Add  a = Mk_Add a a
data HRep_Mul  a = Mk_Mul a a
\end{code}

For this purpose, we employ Haskell's type classes mechanism \tocite{}.
%
We define an evaluation type class $(\Downarrow)$ which specifies how to perform
a single transformation |step| from the higher-order representation back into a
concrete value of the target:

\begin{code}
class (rep down target) where
  step :: rep target -> target
\end{code}

Then, we can define the following overloaded instances of the |step| operation
for the canonical representations of data constructors simply by mapping each
lifted constructor back into its corresponding one.

\begin{code}
instance (HRep_Val down Exp) where
  step (Mk_Val n) = Val n

instance (HRep_Add down Exp) where
  step (Mk_Add x y) = Add x y

instance (HRep_Mul down Exp) where
  step (Mk_Mul x y) = Mul x y
\end{code}

With these individual representations, we can define a type combinator
$(\oplus)$ to compose two representations into a single one:

\begin{code}
data (f oplus g) a = L (f a) | R (g a)
\end{code}

Then, a composite representation can be transformed back into the concrete
target type by pattern matching on the data type variant and applying the |step|
tranformation to the inner representation:

\begin{code}
instance (f oplus g down Exp) where
  step (L f) = step f
  step (R g) = step g
\end{code}

Furthermore, we can define a type combinator ($\otimes$) to tag every
representation with an explicit generation frequency:

\begin{code}
data (f otimes n) a = Freq (f a)
\end{code}

This combinator is evaluated back to our target data type simply by piping the
result from the inner representation.
%
It does not change the evaluation semantics, as it is only considered at
generation time:

\begin{code}
instance (f otimes n down Exp) where
  step (Freq f) = step f
\end{code}

With the introduced combinators, we can easily create a type synonym |HRep_Exp|
to refer to the canonical representation of our original data type |Exp|,
tagging for instance the representation of the constuctor |Add| to be generated
in double the proportion of rest of the representations:

\begin{code}
type HRep_Exp  =       HRep_Val
               oplus'  HRep_Add  otimes 2
               oplus'  HRep_Mul
\end{code}


\subsection{\textbf{Representing Pattern Matchings}}

\begin{code}
data HRep_foo_1  a = Mk_foo_1 a a
data HRep_foo_2  a = Mk_foo_2 Int a
\end{code}

\begin{code}
instance (HRep_foo_1 down Exp) where
  step (Mk_foo_1 x y)
    = Add (Add x (Val 50)) (Add (Val 25) y)

instance (HRep_foo_2 down Exp) where
  step (Mk_foo_2 x y)
    = Mul (Val 50) (Mul (Val x) y)
\end{code}

\begin{code}
type HRep_foo  =       HRep_foo_1
               oplus'  HRep_foo_2
\end{code}

\subsection{\textbf{Representing Abstract Interfaces}}

\begin{code}
data HRep_ten       a = Mk_ten
data HRep_square    a = Mk_square   a
data HRep_minus     a = Mk_minus    a a
\end{code}

\begin{code}
instance (HRep_ten down Exp) where
  step Mk_ten = ten

instance (HRep_square down Exp) where
  step (Mk_square x) = square x

instance (HRep_minus down Exp) where
  step (Mk_minus x y) = minus x y
\end{code}

\begin{code}
type HRep_M  =       HRep_ten
             oplus'  HRep_square
             oplus'  HRep_minus
\end{code}



\begin{code}
type Spec_Exp  =       HRep_Exp  otimes 4
               oplus'  HRep_foo  otimes 2
               oplus'  HRep_M
\end{code}

This previous definition can be interpreted graphically as it is shown in the
Figure \ref{fig:hrep}.
%
Curly arrows represent the structural information extracted using
meta-programming.

\begin{figure}[t]
  \centering
  \section{Extracting Structure} \label{sec:hrep}

In this work, we ...

\begin{code}
data HRep_Val  a = Mk_Val Int
data HRep_Add  a = Mk_Add a a
data HRep_Mul  a = Mk_Mul a a
\end{code}

For this purpose, we employ Haskell's type classes mechanism \tocite{}.
%
We define an evaluation type class $(\Downarrow)$ which specifies how to perform
a single transformation |step| from the higher-order representation back into a
concrete value of the target:

\begin{code}
class (rep down target) where
  step :: rep target -> target
\end{code}

Then, we can define the following overloaded instances of the |step| operation
for the canonical representations of data constructors simply by mapping each
lifted constructor back into its corresponding one.

\begin{code}
instance (HRep_Val down Exp) where
  step (Mk_Val n) = Val n

instance (HRep_Add down Exp) where
  step (Mk_Add x y) = Add x y

instance (HRep_Mul down Exp) where
  step (Mk_Mul x y) = Mul x y
\end{code}

With these individual representations, we can define a type combinator
$(\oplus)$ to compose two representations into a single one:

\begin{code}
data (f oplus g) a = L (f a) | R (g a)
\end{code}

Then, a composite representation can be transformed back into the concrete
target type by pattern matching on the data type variant and applying the |step|
tranformation to the inner representation:

\begin{code}
instance (f oplus g down Exp) where
  step (L f) = step f
  step (R g) = step g
\end{code}

Furthermore, we can define a type combinator ($\otimes$) to tag every
representation with an explicit generation frequency:

\begin{code}
data (f otimes n) a = Freq (f a)
\end{code}

This combinator is evaluated back to our target data type simply by piping the
result from the inner representation.
%
It does not change the evaluation semantics, as it is only considered at
generation time:

\begin{code}
instance (f otimes n down Exp) where
  step (Freq f) = step f
\end{code}

With the introduced combinators, we can easily create a type synonym |HRep_Exp|
to refer to the canonical representation of our original data type |Exp|,
tagging for instance the representation of the constuctor |Add| to be generated
in double the proportion of rest of the representations:

\begin{code}
type HRep_Exp  =       HRep_Val
               oplus'  HRep_Add  otimes 2
               oplus'  HRep_Mul
\end{code}


\subsection{\textbf{Representing Pattern Matchings}}

\begin{code}
data HRep_foo_1  a = Mk_foo_1 a a
data HRep_foo_2  a = Mk_foo_2 Int a
\end{code}

\begin{code}
instance (HRep_foo_1 down Exp) where
  step (Mk_foo_1 x y)
    = Add (Add x (Val 50)) (Add (Val 25) y)

instance (HRep_foo_2 down Exp) where
  step (Mk_foo_2 x y)
    = Mul (Val 50) (Mul (Val x) y)
\end{code}

\begin{code}
type HRep_foo  =       HRep_foo_1
               oplus'  HRep_foo_2
\end{code}

\subsection{\textbf{Representing Abstract Interfaces}}

\begin{code}
data HRep_ten       a = Mk_ten
data HRep_square    a = Mk_square   a
data HRep_minus     a = Mk_minus    a a
\end{code}

\begin{code}
instance (HRep_ten down Exp) where
  step Mk_ten = ten

instance (HRep_square down Exp) where
  step (Mk_square x) = square x

instance (HRep_minus down Exp) where
  step (Mk_minus x y) = minus x y
\end{code}

\begin{code}
type HRep_M  =       HRep_ten
             oplus'  HRep_square
             oplus'  HRep_minus
\end{code}



\begin{code}
type Spec_Exp  =       HRep_Exp  otimes 4
               oplus'  HRep_foo  otimes 2
               oplus'  HRep_M
\end{code}

This previous definition can be interpreted graphically as it is shown in the
Figure \ref{fig:hrep}.
%
Curly arrows represent the structural information extracted using
meta-programming.

\begin{figure}[t]
  \centering
  \input{tikz/hrep.tex}
  \caption{Higher order representation of the data type |Exp|, using structural
    information from the function |foo| and the abstract interface of the module
    |M|.}
  \label{fig:hrep}
\end{figure}

% \begin{code}
% data (Term f) a = TagTerm (f a)

% instance (Term f down Exp) where
%   step (TagTerm f) = step f
% \end{code}
  \caption{Higher order representation of the data type |Exp|, using structural
    information from the function |foo| and the abstract interface of the module
    |M|.}
  \label{fig:hrep}
\end{figure}

% \begin{code}
% data (Term f) a = TagTerm (f a)

% instance (Term f down Exp) where
%   step (TagTerm f) = step f
% \end{code}
  \caption{Higher order representation of the data type |Exp|, using structural
    information from the function |foo| and the abstract interface of the module
    |M|.}
  \label{fig:hrep}
\end{figure}

% \begin{code}
% data (Term f) a = TagTerm (f a)

% instance (Term f down Exp) where
%   step (TagTerm f) = step f
% \end{code}
  \caption{The idea}
\end{figure}

Pellentesque dapibus suscipit ligula. Donec posuere augue in quam. Etiam vel
tortor sodales tellus ultricies commodo. Suspendisse potenti. Aenean in sem ac
leo mollis blandit. Donec neque quam, dignissim in, mollis nec, sagittis eu,
wisi. Phasellus lacus. Etiam laoreet quam sed arcu. Phasellus at dui in ligula
mollis ultricies. Integer placerat tristique nisl. Praesent augue. Fusce
commodo. Vestibulum convallis, lorem a tempus semper, dui dui euismod elit,
vitae placerat urna tortor vitae lacus. Nullam libero mauris, consequat quis,
varius et, dictum id, arcu. Mauris mollis tincidunt felis. Aliquam feugiat
tellus ut neque. Nulla facilisis, risus a rhoncus fermentum, tellus tellus
lacinia purus, et dictum nunc justo sit amet elit.

Nullam eu ante vel est convallis dignissim. Fusce suscipit, wisi nec facilisis
facilisis, est dui fermentum leo, quis tempor ligula erat quis odio. Nunc porta
vulputate tellus. Nunc rutrum turpis sed pede. Sed bibendum. Aliquam posuere.
Nunc aliquet, augue nec adipiscing interdum, lacus tellus malesuada massa, quis
varius mi purus non odio. Pellentesque condimentum, magna ut suscipit hendrerit,
ipsum augue ornare nulla, non luctus diam neque sit amet urna. Curabitur
vulputate vestibulum lorem. Fusce sagittis, libero non molestie mollis, magna
orci ultrices dolor, at vulputate neque nulla lacinia eros. Sed id ligula quis
est convallis tempor. Curabitur lacinia pulvinar nibh. Nam a sapien.
\section{Random Generators Synthesis} \label{sec:synthesis}

Aliquam erat volutpat. Nunc eleifend leo vitae magna. In id erat non orci
commodo lobortis. Proin neque massa, cursus ut, gravida ut, lobortis eget,
lacus. Sed diam. Praesent fermentum tempor tellus. Nullam tempus. Mauris ac
felis vel velit tristique imperdiet. Donec at pede. Etiam vel neque nec dui
dignissim bibendum. Vivamus id enim. Phasellus neque orci, porta a, aliquet
quis, semper a, massa. Phasellus purus. Pellentesque tristique imperdiet tortor.
Nam euismod tellus id erat.

Nullam eu ante vel est convallis dignissim. Fusce suscipit, wisi nec facilisis
facilisis, est dui fermentum leo, quis tempor ligula erat quis odio. Nunc porta
vulputate tellus. Nunc rutrum turpis sed pede. Sed bibendum. Aliquam posuere.
Nunc aliquet, augue nec adipiscing interdum, lacus tellus malesuada massa, quis
varius mi purus non odio. Pellentesque condimentum, magna ut suscipit hendrerit,
ipsum augue ornare nulla, non luctus diam neque sit amet urna. Curabitur
vulputate vestibulum lorem. Fusce sagittis, libero non molestie mollis, magna
orci ultrices dolor, at vulputate neque nulla lacinia eros. Sed id ligula quis
est convallis tempor. Curabitur lacinia pulvinar nibh. Nam a sapien.

Lorem ipsum dolor sit amet, consectetuer adipiscing elit. Donec hendrerit tempor
tellus. Donec pretium posuere tellus. Proin quam nisl, tincidunt et, mattis
eget, convallis nec, purus. Cum sociis natoque penatibus et magnis dis
parturient montes, nascetur ridiculus mus. Nulla posuere. Donec vitae dolor.
Nullam tristique diam non turpis. Cras placerat accumsan nulla. Nullam rutrum.
Nam vestibulum accumsan nisl.
\section{Case Studies}

Aliquam erat volutpat. Nunc eleifend leo vitae magna. In id erat non orci
commodo lobortis. Proin neque massa, cursus ut, gravida ut, lobortis eget,
lacus. Sed diam. Praesent fermentum tempor tellus. Nullam tempus. Mauris ac
felis vel velit tristique imperdiet. Donec at pede. Etiam vel neque nec dui
dignissim bibendum. Vivamus id enim. Phasellus neque orci, porta a, aliquet
quis, semper a, massa. Phasellus purus. Pellentesque tristique imperdiet tortor.
Nam euismod tellus id erat.

Aliquam erat volutpat. Nunc eleifend leo vitae magna. In id erat non orci
commodo lobortis. Proin neque massa, cursus ut, gravida ut, lobortis eget,
lacus. Sed diam. Praesent fermentum tempor tellus. Nullam tempus. Mauris ac
felis vel velit tristique imperdiet. Donec at pede. Etiam vel neque nec dui
dignissim bibendum. Vivamus id enim. Phasellus neque orci, porta a, aliquet
quis, semper a, massa. Phasellus purus. Pellentesque tristique imperdiet tortor.
Nam euismod tellus id erat.

Lorem ipsum dolor sit amet, consectetuer adipiscing elit. Donec hendrerit tempor
tellus. Donec pretium posuere tellus. Proin quam nisl, tincidunt et, mattis
eget, convallis nec, purus. Cum sociis natoque penatibus et magnis dis
parturient montes, nascetur ridiculus mus. Nulla posuere. Donec vitae dolor.
Nullam tristique diam non turpis. Cras placerat accumsan nulla. Nullam rutrum.
Nam vestibulum accumsan nisl.

Nullam eu ante vel est convallis dignissim. Fusce suscipit, wisi nec facilisis
facilisis, est dui fermentum leo, quis tempor ligula erat quis odio. Nunc porta
vulputate tellus. Nunc rutrum turpis sed pede. Sed bibendum. Aliquam posuere.
Nunc aliquet, augue nec adipiscing interdum, lacus tellus malesuada massa, quis
varius mi purus non odio. Pellentesque condimentum, magna ut suscipit hendrerit,
ipsum augue ornare nulla, non luctus diam neque sit amet urna. Curabitur
vulputate vestibulum lorem. Fusce sagittis, libero non molestie mollis, magna
orci ultrices dolor, at vulputate neque nulla lacinia eros. Sed id ligula quis
est convallis tempor. Curabitur lacinia pulvinar nibh. Nam a sapien.
\section{Related Work}

Aliquam erat volutpat. Nunc eleifend leo vitae magna. In id erat non orci
commodo lobortis. Proin neque massa, cursus ut, gravida ut, lobortis eget,
lacus. Sed diam. Praesent fermentum tempor tellus. Nullam tempus. Mauris ac
felis vel velit tristique imperdiet. Donec at pede. Etiam vel neque nec dui
dignissim bibendum. Vivamus id enim. Phasellus neque orci, porta a, aliquet
quis, semper a, massa. Phasellus purus. Pellentesque tristique imperdiet tortor.
Nam euismod tellus id erat.

Lorem ipsum dolor sit amet, consectetuer adipiscing elit. Donec hendrerit tempor
tellus. Donec pretium posuere tellus. Proin quam nisl, tincidunt et, mattis
eget, convallis nec, purus. Cum sociis natoque penatibus et magnis dis
parturient montes, nascetur ridiculus mus. Nulla posuere. Donec vitae dolor.
Nullam tristique diam non turpis. Cras placerat accumsan nulla. Nullam rutrum.
Nam vestibulum accumsan nisl.

Aliquam erat volutpat. Nunc eleifend leo vitae magna. In id erat non orci
commodo lobortis. Proin neque massa, cursus ut, gravida ut, lobortis eget,
lacus. Sed diam. Praesent fermentum tempor tellus. Nullam tempus. Mauris ac
felis vel velit tristique imperdiet. Donec at pede. Etiam vel neque nec dui
dignissim bibendum. Vivamus id enim. Phasellus neque orci, porta a, aliquet
quis, semper a, massa. Phasellus purus. Pellentesque tristique imperdiet tortor.
Nam euismod tellus id erat.

Aliquam erat volutpat. Nunc eleifend leo vitae magna. In id erat non orci
commodo lobortis. Proin neque massa, cursus ut, gravida ut, lobortis eget,
lacus. Sed diam. Praesent fermentum tempor tellus. Nullam tempus. Mauris ac
felis vel velit tristique imperdiet. Donec at pede. Etiam vel neque nec dui
dignissim bibendum. Vivamus id enim. Phasellus neque orci, porta a, aliquet
quis, semper a, massa. Phasellus purus. Pellentesque tristique imperdiet tortor.
Nam euismod tellus id erat.
\section{Final Remarks}

Nullam eu ante vel est convallis dignissim. Fusce suscipit, wisi nec facilisis
facilisis, est dui fermentum leo, quis tempor ligula erat quis odio. Nunc porta
vulputate tellus. Nunc rutrum turpis sed pede. Sed bibendum. Aliquam posuere.
Nunc aliquet, augue nec adipiscing interdum, lacus tellus malesuada massa, quis
varius mi purus non odio. Pellentesque condimentum, magna ut suscipit hendrerit,
ipsum augue ornare nulla, non luctus diam neque sit amet urna. Curabitur
vulputate vestibulum lorem. Fusce sagittis, libero non molestie mollis, magna
orci ultrices dolor, at vulputate neque nulla lacinia eros. Sed id ligula quis
est convallis tempor. Curabitur lacinia pulvinar nibh. Nam a sapien.

Nullam eu ante vel est convallis dignissim. Fusce suscipit, wisi nec facilisis
facilisis, est dui fermentum leo, quis tempor ligula erat quis odio. Nunc porta
vulputate tellus. Nunc rutrum turpis sed pede. Sed bibendum. Aliquam posuere.
Nunc aliquet, augue nec adipiscing interdum, lacus tellus malesuada massa, quis
varius mi purus non odio. Pellentesque condimentum, magna ut suscipit hendrerit,
ipsum augue ornare nulla, non luctus diam neque sit amet urna. Curabitur
vulputate vestibulum lorem. Fusce sagittis, libero non molestie mollis, magna
orci ultrices dolor, at vulputate neque nulla lacinia eros. Sed id ligula quis
est convallis tempor. Curabitur lacinia pulvinar nibh. Nam a sapien.

Aliquam erat volutpat. Nunc eleifend leo vitae magna. In id erat non orci
commodo lobortis. Proin neque massa, cursus ut, gravida ut, lobortis eget,
lacus. Sed diam. Praesent fermentum tempor tellus. Nullam tempus. Mauris ac
felis vel velit tristique imperdiet. Donec at pede. Etiam vel neque nec dui
dignissim bibendum. Vivamus id enim. Phasellus neque orci, porta a, aliquet
quis, semper a, massa. Phasellus purus. Pellentesque tristique imperdiet tortor.
Nam euismod tellus id erat.

%%%%%%%%%%%%%%%%%%%%%%%%%%%%%%%%%%%%%%%%
%% Bibliography
\bibliographystyle{IEEEtran}
\bibliography{references.bib}

\end{document}
