\documentclass[conference, fleqn]{IEEEtran}

%include polycode.fmt
%include notation.fmt

\IEEEoverridecommandlockouts
% The preceding line is only needed to identify funding in the first footnote. If that is unneeded, please comment it out.

\usepackage{todonotes}
\usepackage{cite}
\usepackage{amsmath,amssymb,amsfonts}
\usepackage{algorithmic}
\usepackage{graphicx}
\usepackage{textcomp}
\usepackage{xcolor}
\usepackage{mathdots}
\usepackage{yhmath}
\usepackage{cancel}
\usepackage{color}
\usepackage{siunitx}
\usepackage{array}
\usepackage{multirow}
\usepackage{amssymb}
\usepackage{gensymb}
\usepackage{tabularx}
\usepackage{wrapfig}
\usepackage{booktabs}
\usepackage{caption}
\usepackage{xspace}
\usepackage{relsize}
\usepackage{bm}
\usepackage{microtype}
\usepackage{tikz}
\usepackage{hyperref}
\usepackage[para]{footmisc}
\usepackage{pgfplots}
\usetikzlibrary{bending,patterns,matrix,fadings,positioning,shapes}
\pgfplotsset{compat=1.12}


%%%%%%%%%%%%%%%%%%%%%%%%%%%%%%%%%%%%%%%%
%% Colors
\definecolor{darkred}{HTML}{600018}
\definecolor{darkblue}{HTML}{1D2F73}
\definecolor{darkgreen}{HTML}{417505}
\definecolor{lightblue}{HTML}{76E9BB}

%%%%%%%%%%%%%%%%%%%%%%%%%%%%%%%%%%%%%%%%
%% Macros
\newcommand{\tocite}{\textbf{CITE}}
\newcommand{\quickcheck}{\emph{QuickCheck}\xspace}
\newcommand{\megadeth}{\emph{MegaDeTH}\xspace}
\newcommand{\dragen}{\dragenlogo\xspace}
\newcommand{\dragenp}{\dragenplogo\xspace}

%% To see the notes
\paperwidth=\dimexpr \paperwidth + 6cm\relax
\oddsidemargin=\dimexpr\oddsidemargin + 3cm\relax
\evensidemargin=\dimexpr\evensidemargin + 3cm\relax
\marginparwidth=\dimexpr \marginparwidth + 3cm\relax
\newcommand{\todoale}[1]{\todo[color=blue!40]{#1}}
%%%%

\newlength\htG\newlength\dpg
\protected\def\dragenlogo{\settoheight{\htG}{G}\settodepth{\dpg}{g}%
  \raisebox{-0.0\dpg}{\includegraphics[height=0.6\htG+\dpg]{dragenlogo}}}
\protected\def\dragenplogo{\settoheight{\htG}{G}\settodepth{\dpg}{g}%
  \raisebox{-0.0\dpg}{\includegraphics[height=0.6\htG+\dpg]{dragenplogo}}}



%%%%%%%%%%%%%%%%%%%%%%%%%%%%%%%%%%%%%%%%
%% Some parameters
\setlength{\mathindent}{\parindent}

\begin{document}

%%%%%%%%%%%%%%%%%%%%%%%%%%%%%%%%%%%%%%%%
%% Title
\title{Generating Structurally Rich \\ Random Algebraic Data Type Values}

%%%%%%%%%%%%%%%%%%%%%%%%%%%%%%%%%%%%%%%%
%% Authors
\author{
  \IEEEauthorblockN{
    % 1\textsuperscript{st}
    Agust\'in Mista
  }
  \IEEEauthorblockA{
    % \textit{Department of Computer Science and Engineering} \\
    \textit{Chalmers University of Technology}\\
    Gothenburg, Sweden \\
    mista@@chalmers.se
  }
\and
\IEEEauthorblockN{
  % 2\textsuperscript{nd}
  Alejandro Russo
  }
  \IEEEauthorblockA{
    % \textit{Department of Computer Science and Engineering} \\
    \textit{Chalmers University of Technology}\\
    Gothenburg, Sweden \\
    russo@@chalmers.se
  }
}

\maketitle


\newenvironment{CompactItemize}%
  {\begin{list}{$\, \  \blacktriangleright$}%
   {\leftmargin=0pt \itemsep=2pt \topsep=5pt
     \parsep=0pt \partopsep=0pt}}%
  {\end{list}}


%%%%%%%%%%%%%%%%%%%%%%%%%%%%%%%%%%%%%%%%
%% Abstract

%include abstract.tex

%%%%%%%%%%%%%%%%%%%%%%%%%%%%%%%%%%%%%%%%
%% Keywords
\begin{IEEEkeywords}
random testing, Haskell, meta-programming
\end{IEEEkeywords}

%%%%%%%%%%%%%%%%%%%%%%%%%%%%%%%%%%%%%%%%
%% Sections

%include introduction.tex
%\newpage

%include randomtesting.tex

%include hrep.tex

%include hrep2.tex

%include synthesis.tex

%include casestudies.tex

%include relatedwork.tex

%include finalremarks.tex

%%%%%%%%%%%%%%%%%%%%%%%%%%%%%%%%%%%%%%%%
%% Bibliography
\bibliographystyle{IEEEtran}
\bibliography{references.bib}

\end{document}

% Local Variables:
% TeX-master: "main.lhs.tex"
% TeX-command-default: "Make"
% End:
